\PassOptionsToPackage{unicode=true}{hyperref} % options for packages loaded elsewhere
\PassOptionsToPackage{hyphens}{url}
%
\documentclass[]{article}
\usepackage{lmodern}
\usepackage{amssymb,amsmath}
\usepackage{ifxetex,ifluatex}
\usepackage{fixltx2e} % provides \textsubscript
\ifnum 0\ifxetex 1\fi\ifluatex 1\fi=0 % if pdftex
  \usepackage[T1]{fontenc}
  \usepackage[utf8]{inputenc}
  \usepackage{textcomp} % provides euro and other symbols
\else % if luatex or xelatex
  \usepackage{unicode-math}
  \defaultfontfeatures{Ligatures=TeX,Scale=MatchLowercase}
\fi
% use upquote if available, for straight quotes in verbatim environments
\IfFileExists{upquote.sty}{\usepackage{upquote}}{}
% use microtype if available
\IfFileExists{microtype.sty}{%
\usepackage[]{microtype}
\UseMicrotypeSet[protrusion]{basicmath} % disable protrusion for tt fonts
}{}
\IfFileExists{parskip.sty}{%
\usepackage{parskip}
}{% else
\setlength{\parindent}{0pt}
\setlength{\parskip}{6pt plus 2pt minus 1pt}
}
\usepackage{hyperref}
\hypersetup{
            pdfborder={0 0 0},
            breaklinks=true}
\urlstyle{same}  % don't use monospace font for urls
\usepackage[a4paper]{geometry}
\usepackage{color}
\usepackage{fancyvrb}
\newcommand{\VerbBar}{|}
\newcommand{\VERB}{\Verb[commandchars=\\\{\}]}
\DefineVerbatimEnvironment{Highlighting}{Verbatim}{commandchars=\\\{\}}
% Add ',fontsize=\small' for more characters per line
\newenvironment{Shaded}{}{}
\newcommand{\AlertTok}[1]{\textcolor[rgb]{1.00,0.00,0.00}{\textbf{#1}}}
\newcommand{\AnnotationTok}[1]{\textcolor[rgb]{0.38,0.63,0.69}{\textbf{\textit{#1}}}}
\newcommand{\AttributeTok}[1]{\textcolor[rgb]{0.49,0.56,0.16}{#1}}
\newcommand{\BaseNTok}[1]{\textcolor[rgb]{0.25,0.63,0.44}{#1}}
\newcommand{\BuiltInTok}[1]{#1}
\newcommand{\CharTok}[1]{\textcolor[rgb]{0.25,0.44,0.63}{#1}}
\newcommand{\CommentTok}[1]{\textcolor[rgb]{0.38,0.63,0.69}{\textit{#1}}}
\newcommand{\CommentVarTok}[1]{\textcolor[rgb]{0.38,0.63,0.69}{\textbf{\textit{#1}}}}
\newcommand{\ConstantTok}[1]{\textcolor[rgb]{0.53,0.00,0.00}{#1}}
\newcommand{\ControlFlowTok}[1]{\textcolor[rgb]{0.00,0.44,0.13}{\textbf{#1}}}
\newcommand{\DataTypeTok}[1]{\textcolor[rgb]{0.56,0.13,0.00}{#1}}
\newcommand{\DecValTok}[1]{\textcolor[rgb]{0.25,0.63,0.44}{#1}}
\newcommand{\DocumentationTok}[1]{\textcolor[rgb]{0.73,0.13,0.13}{\textit{#1}}}
\newcommand{\ErrorTok}[1]{\textcolor[rgb]{1.00,0.00,0.00}{\textbf{#1}}}
\newcommand{\ExtensionTok}[1]{#1}
\newcommand{\FloatTok}[1]{\textcolor[rgb]{0.25,0.63,0.44}{#1}}
\newcommand{\FunctionTok}[1]{\textcolor[rgb]{0.02,0.16,0.49}{#1}}
\newcommand{\ImportTok}[1]{#1}
\newcommand{\InformationTok}[1]{\textcolor[rgb]{0.38,0.63,0.69}{\textbf{\textit{#1}}}}
\newcommand{\KeywordTok}[1]{\textcolor[rgb]{0.00,0.44,0.13}{\textbf{#1}}}
\newcommand{\NormalTok}[1]{#1}
\newcommand{\OperatorTok}[1]{\textcolor[rgb]{0.40,0.40,0.40}{#1}}
\newcommand{\OtherTok}[1]{\textcolor[rgb]{0.00,0.44,0.13}{#1}}
\newcommand{\PreprocessorTok}[1]{\textcolor[rgb]{0.74,0.48,0.00}{#1}}
\newcommand{\RegionMarkerTok}[1]{#1}
\newcommand{\SpecialCharTok}[1]{\textcolor[rgb]{0.25,0.44,0.63}{#1}}
\newcommand{\SpecialStringTok}[1]{\textcolor[rgb]{0.73,0.40,0.53}{#1}}
\newcommand{\StringTok}[1]{\textcolor[rgb]{0.25,0.44,0.63}{#1}}
\newcommand{\VariableTok}[1]{\textcolor[rgb]{0.10,0.09,0.49}{#1}}
\newcommand{\VerbatimStringTok}[1]{\textcolor[rgb]{0.25,0.44,0.63}{#1}}
\newcommand{\WarningTok}[1]{\textcolor[rgb]{0.38,0.63,0.69}{\textbf{\textit{#1}}}}
\usepackage{longtable,booktabs}
% Fix footnotes in tables (requires footnote package)
\IfFileExists{footnote.sty}{\usepackage{footnote}\makesavenoteenv{longtable}}{}
\setlength{\emergencystretch}{3em}  % prevent overfull lines
\providecommand{\tightlist}{%
  \setlength{\itemsep}{0pt}\setlength{\parskip}{0pt}}
\setcounter{secnumdepth}{0}
% Redefines (sub)paragraphs to behave more like sections
\ifx\paragraph\undefined\else
\let\oldparagraph\paragraph
\renewcommand{\paragraph}[1]{\oldparagraph{#1}\mbox{}}
\fi
\ifx\subparagraph\undefined\else
\let\oldsubparagraph\subparagraph
\renewcommand{\subparagraph}[1]{\oldsubparagraph{#1}\mbox{}}
\fi

% set default figure placement to htbp
\makeatletter
\def\fps@figure{htbp}
\makeatother


\date{}

\begin{document}

\textbf{UNFINISHED - WORK IN PROGRESS }

** Torsor **

This is a very small C++ one-header one-class library for expressing and
enforcing the difference between relative and absolute (anchored)
values, aka torsors. Much like a unit system, this library uses the type
system to help you prevent some erroneous operations. It can also help
to make an API simpler and more elegant. By design, it has no runtime
impact.

\hypertarget{author}{%
\subsubsection{Author}\label{author}}

\begin{verbatim}
+ Wouter van Ooijen (wouter@voti.nl)
\end{verbatim}

\hypertarget{licneses}{%
\subsubsection{Licneses}\label{licneses}}

\begin{verbatim}
+ souyrce license [boost](https://www.boost.org/users/license.html)
+ documentation license (this file and te other .md files): [CC BY-SA 2.5](
\end{verbatim}

https://creativecommons.org/licenses/by-sa/2.5/)

\hypertarget{requires}{%
\subsubsection{Requires}\label{requires}}

\begin{verbatim}
+ gcc > 6.2 (I used GCC/MinGW 7.3.0) with -fconcepts
\end{verbatim}

\begin{center}\rule{0.5\linewidth}{\linethickness}\end{center}

\hypertarget{introduction}{%
\section{Introduction}\label{introduction}}

We are used to numerical value types that can be added, subtracted,
multiplied, and divided. But for some real-world values only a more
limited set of operations make sense.

The most illustrating example is perhaps time (as measured in some unit,
let's assume seconds). There are two subtly distinct notions of time:

\begin{itemize}
\tightlist
\item
  a duration (hwo long somehing took)
\item
  a moment in time (when something happened)
\end{itemize}

It makes sense to add two durations (10 seconds + 5 seconds = 15
seconds) but it makes no sense to add two moments, like today and
tomorrow, or now and 10 minutes ago. Just as adding meters to seconds
doesn't make any sense, adding two time moments doesn't make any sense.
Subtracting two moments on the other hand does make sense, but the
result is a duration, rather than a moment in time.

In terms of scales, for a value type that denotes a ratio scale value (a
value for which addition yields a value on the same scale), the torsor
of that type is the corresponding interval scale (anchored) type.

In mathematical terms, the set of moments in time is the \emph{torsor}
of the set of time durations.

Examples of ratio scales and their corresponding torsors (anchored
interval scales) are:

\begin{longtable}[]{@{}ll@{}}
\toprule
ratio scale \textbf{T} & interval scale \textbf{torsor\textless{} T
\textgreater{}}\tabularnewline
\midrule
\endhead
duration & moment in time\tabularnewline
temperature difference & absolute temperature\tabularnewline
distance vector & location\tabularnewline
\bottomrule
\end{longtable}

Whether a scale is a torsor or not has nothing to do with its unit: in a
unit system like SI a ratio type and its torsor have the same unit.

But just like adding two values that have different SI units makes no
sense, adding two torsor values makes no sense. The torsor class
template uses the type system to block such meaningless operations at
compile time. It is designed to have zero runtime overhead.

Having different types for a ration scale and its torsor can make an API
more elegant, because it makes the difference explicit in the type
system, instead of requiring functions with different names.

\begin{center}\rule{0.5\linewidth}{\linethickness}\end{center}

\hypertarget{mathematical-background}{%
\section{Mathematical background}\label{mathematical-background}}

As I understand it, a torsor is a mathematical abstraction over a group
(a set of values with associated operations) that assigns a special
meaning to one value.

\begin{itemize}
\item
  The
  \href{https://en.wikipedia.org/wiki/Torsor_(algebraic_geometry)}{torsor
  wiki} is not very readable for a non-mathematician.
\item
  This \href{http://math.ucr.edu/home/baez/torsors.html}{Torsors Made
  Easy} page is quite readable.
\item
  This
  \href{https://golem.ph.utexas.edu/category/2013/06/torsors_and_enriched_categorie.html}{blog
  from The n-Category Cafe} tries to be accessible, but I guess I am not
  part of the intended audience.
\item
  \href{https://ncatlab.org/nlab/show/torsor}{This page} gives the
  slogan ``A torsor is like a group that has forgotten its neutral
  element.'' Otherwise I found it hard to read.
\end{itemize}

\begin{center}\rule{0.5\linewidth}{\linethickness}\end{center}

\hypertarget{interface-summary}{%
\section{Interface summary}\label{interface-summary}}

In simple terms: with a torsor you can

\begin{itemize}
\tightlist
\item
  construct, copy, assign
\item
  add and subtract things that can be added to or subtracted from its
  base
\item
  compare torsors, which means comparing their base values
\item
  subtract two torsors, which yields its base type
\end{itemize}

More formally: the library provides a final class template
\emph{torsor\textless{}typename B\textgreater{}}. The type B must have a
constructor that can be called with a single value 0.

The library supports the following operations:

\begin{itemize}
\item
  default constructor, copy constructor, assignment operator
\item
  for each torsor\textless{} B \textgreater{} t and X x: ( t + x ), ( x
  + t ), ( t - x ), ( x - t )

  These operators are provided if and only if they are available for B
  and X. The result is a torsor of the decltype( t op x ) or ( x op t).
\item
  for each torsor\textless{} B \textgreater{} t and X x: ( t += x ), ( t
  -= x )

  These operators are provided if and only if they are available for B
  and X. The result is a reference to the (appropriately modified) t.
\item
  for each torsor\textless{} B \textgreater{} b and torsor\textless{} C
  \textgreater{} c: ( b \textgreater{} c ), ( b \textgreater{}= c ), ( b
  \textless{} c ), ( b \textless{}= c ), ( b == c ), ( b != c )

  These operators are provided if and only if they are available for B
  and C. The result is the result of the same comparison on the base
  value: ( t op c ) or (c op t ).
\end{itemize}

All operators are const and constexpr, where appropriate. There are
currently no exception annotations (I work with -fno-exceptions). The
library itself doen't generate any exceptions, but the operations it
does with the base type could.

\begin{center}\rule{0.5\linewidth}{\linethickness}\end{center}

\hypertarget{use}{%
\section{Use}\label{use}}

The library is the single header file library/torsor.hpp, so you can
copy it to some suitable place (where your compiler can find it) and
insert

\begin{Shaded}
\begin{Highlighting}[]
\PreprocessorTok{#include }\ImportTok{<torsor.hpp>}
\end{Highlighting}
\end{Shaded}

in your source file(s).

The library uses concepts, so GCC 6.2 or later is required, with the
\emph{-fconcepts} command-line flag.

The torsor.hpp file contains Doxygen comments. The command ''`make
docs''' in the root directory generates the Doxygen pages (Doxygen
required), and a pdf version of this file (pandoc required).

\begin{center}\rule{0.5\linewidth}{\linethickness}\end{center}

\hypertarget{usage-example-timing}{%
\section{Usage example: timing}\label{usage-example-timing}}

Imagine a timing library for a small embedded system that uses the type
\emph{duration} to express an amount of time. It has a function
\emph{now()} that returns the current time, expressed as the time
elapsed since some (undefined) epoch. (For an embedded system the epoch
could be the moment the system was last switched on.) The type returned
by \emph{now()} should be
\emph{torsor\textless{}duration\textgreater{}}, because it is not a
duration, but the difference between two such values \emph{is} a
duration.

\begin{Shaded}
\begin{Highlighting}[]
\KeywordTok{using}\NormalTok{ duration = ...}
\NormalTok{torsor< duration > now();}
\end{Highlighting}
\end{Shaded}

Our timing library is likely to have a function that can be called to
wait some time. In fact, it will likely have two such functions: one
that takes and amount of time as argument and waits for that amount of
time, and one that takes a moment in time as argument, and waits until
that moment has arrived. The argument of the first function is a
\emph{duration} (an amount of time), the argument of the second function
is a \emph{moment} in time. Making the argument of the second function a
torsor the two functions can be overloaded.

\begin{Shaded}
\begin{Highlighting}[]
\CommentTok{// wait for the specified amount of time}
\DataTypeTok{void}\NormalTok{ wait( duration ); }

\CommentTok{// wait until the specified moment in time}
\DataTypeTok{void}\NormalTok{ wait( torsor< duration > );}
\end{Highlighting}
\end{Shaded}

With these definitions a user can't make the mistake of adding two
moments.

\begin{Shaded}
\begin{Highlighting}[]
\KeywordTok{auto}\NormalTok{ a = now();}
\KeywordTok{auto}\NormalTok{ b = now();}
\NormalTok{a + b: }\CommentTok{// won't compile}
\end{Highlighting}
\end{Shaded}

This simple benchmark function shows that the difference between two
moments is a duration.

\begin{Shaded}
\begin{Highlighting}[]
\KeywordTok{template}\NormalTok{< }\KeywordTok{typename}\NormalTok{ F >}
\NormalTok{duration time_to_run( F work )\{}
   \KeywordTok{auto}\NormalTok{ start = now();}
\NormalTok{   work();}
   \KeywordTok{auto}\NormalTok{ stop = now();}
   \ControlFlowTok{return}\NormalTok{ stop - start;}
\NormalTok{\};   }
\end{Highlighting}
\end{Shaded}

The distinction between absolute time and moments in time can be found
in the C++ \href{https://en.cppreference.com/w/cpp/chrono}{std::chrono}
library, but that library does not generalise the concept of a ratio
value range and its corresponding torsor (interval value range).

\begin{center}\rule{0.5\linewidth}{\linethickness}\end{center}

\hypertarget{usage-example-rectangle}{%
\section{Usage example: rectangle}\label{usage-example-rectangle}}

A graphics library will have a type \emph{location} that specifies a
place on the graphics screen. A class that represents a rectangle object
on the screen will take one argument to specify the start (upper left)
point of the rectangle, and one more argument. But what does that second
argument specify;

\begin{itemize}
\item
  the end (lower right) point, or
\item
  the size of the rectangle (the distance between the upper-left and
  lower-right points)?
\end{itemize}

Both are valid choices.

When you realize that a location on the screen is actually the torsor of
a distance on the screen, the two options can be provided by two
constructors that take different second arguments.

\begin{Shaded}
\begin{Highlighting}[]
\KeywordTok{using}\NormalTok{ distance = ...}
\KeywordTok{using}\NormalTok{ location = torsor< distance >;}

\KeywordTok{class}\NormalTok{ rectangle \{}
\NormalTok{   . . .}
\KeywordTok{public}\NormalTok{:}
\NormalTok{   rectangle( location start, location end );}
   
\NormalTok{   rectangle( location start, distance size ): }
\NormalTok{      rectangle( start, start + size )\{\}}
\NormalTok{\}; }
\end{Highlighting}
\end{Shaded}

\begin{center}\rule{0.5\linewidth}{\linethickness}\end{center}

\hypertarget{usage-example-temperature}{%
\section{Usage example: temperature}\label{usage-example-temperature}}

In his \href{https://www.youtube.com/watch?v=nN5ya6oNImg}{talk at ACCU
2019} Mateusz Pusz' asked what (if anything) is the sum of two
temperatures, for instance 10 degrees Celcius + 20 degrees Celcius? He
proposed these possible answers:

\begin{itemize}
\item
  30 degrees Celcius
\item
  303 degrees Celcius
\item
  meaningless
\end{itemize}

He then gave reasonable arguments for all three answers.

With the torsor concept we can unravel this problem by first asking for
more information: are those values to be interpreted as temperature
differences, or as absolute temperatures (the torsor of temperature
differences).

If they are temperature differences we must flog the author and insist
that in the future he writes them as N degrees (without the Celcius),
because in that case X degrees Celcius is the same as X degrees Kelvin,
so appending `Celcius' or `Kelvin' is misleading. (Let's forget about
Fahrenheit and Reaumur.) In that case, 10 degrees + 20 degrees is
without any doubt 30 degrees.

If the values are tosor (absolute) temperatures, adding them is
meaningless, insofar that it doesn't produce an absolute temperature or
a temperature difference.

One could argue that if the addition means anything, it is that it
produces something that:

\begin{itemize}
\item
  when divided by 2, yields an absolute (torsor) temperature
\item
  when you subtract an absolute (torsor) temperature from it, it yields
  an absolute (torsor) temperature.
\end{itemize}

I doubt that is usefull to anyone (but check the note at the end of this
file about averaging).

\begin{center}\rule{0.5\linewidth}{\linethickness}\end{center}

\hypertarget{limitations}{%
\section{Limitations}\label{limitations}}

The torsor class limits the operations on a torsor to adding or
subtracting a base type value, or subtraction two torsors to yield a
base type value. As a colleague remarked, this makes it difficult to
average a number of torsor values, which is a perfectly sensible
operation.

With the current installation, you can't use torsor\textless{}
torsor\textless{} T \textgreater{}\textgreater{}, because a torsor
requires its base type to have a constructor that accepts as a
parameter.

\begin{center}\rule{0.5\linewidth}{\linethickness}\end{center}

\hypertarget{to-do-list}{%
\section{To do list}\label{to-do-list}}

\begin{itemize}
\tightlist
\item
  find a nice torsor picture that isn't a user-defined pokemon
\end{itemize}

\end{document}
